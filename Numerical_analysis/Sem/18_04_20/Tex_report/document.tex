\documentclass[14pt,a4paper]{scrartcl}
\usepackage{cmap}
\usepackage[utf8]{inputenc}
\usepackage[T1,T2A]{fontenc}
\usepackage[english,russian]{babel}
\usepackage{relsize}
\usepackage{graphicx}
\usepackage{subfigure}
\usepackage{mathtools}
\usepackage{amssymb}
\usepackage{float}
\usepackage{sidecap}
\usepackage{wrapfig}
\usepackage{caption}
\usepackage[table,xcdraw]{xcolor}
\usepackage{minted}
\usepackage{tcolorbox}
\usepackage{enumitem}
\makeatletter

\renewcommand{\thesubsection}{\arabic{subsection}}

\newenvironment{sqcases}{%
	\matrix@check\sqcases\env@sqcases
}{%
	\endarray\right.%
}
\def\env@sqcases{%
	\let\@ifnextchar\new@ifnextchar
	\left\lbrack
	\def\arraystretch{1.2}%
	\array{@{}l@{\quad}l@{}}%
}
\makeatother

\begin{document}
	\begin{titlepage}
	\begin{center}
		\large
		МИНИСТЕРСТВО НАУКИ И ВЫСШЕГО ОБРАЗОВАНИЯ\\ РОССИЙСКОЙ ФЕДЕРАЦИИ
		
		\vspace{0.5cm}
		
		МГТУ им Н.Э.Баумана
		\vspace{0.25cm}
		
		Факультет ФН
		
		Кафедра вычислительной математики и математической физики
		\vfill
		
		
		Соколов Арсений Андреевич\\
		\vfill
		
		
		{\LARGE Семинар от 18.04.20 \\ по основам сеточных методов\\[2mm]
		}
		\bigskip
		
		3 курс, группа ФН11-63Б\\
		Вариант 3
	\end{center}
	\vfill
	
	\newlength{\ML}
	\settowidth{\ML}{«\underline{\hspace{0.7cm}}» \underline{\hspace{2cm}}}
	\hfill\begin{minipage}{0.4\textwidth}
		Преподаватель\\
		\underline{\hspace{3cm}} В.\,А.~Кутыркин\\
		«\underline{\hspace{0.7cm}}» \underline{\hspace{1.71cm}} 2020 г.
	\end{minipage}%
	\bigskip
	
	
	\vfill
	
	\begin{center}
		Москва, 2020 г.
	\end{center}
\end{titlepage}

\section*{Задачи для решения на семинаре}

Методом матричной прогонки найти решение СЛАУ ($k$ -- номер группы, $N$ -- номер фамилии студента в журнале группы), сделав проверку:

\begin{equation}\label{1}
	\mathbf{D} \cdot ^>\mathbf{x}_{(\cdot)} = ^>\mathbf{f}_{(\cdot)},
\end{equation}

где $^>\mathbf{f}_{(\cdot)} = \left[ ^>\mathbf{f}_{(1)}, ^>\mathbf{f}_{(2)}, ^>\mathbf{f}_{(3)}, ^>\mathbf{f}_{(4)} \right\rangle \in ^>\left(^>\mathbb{R}^3\right)^4, \quad ^>\mathbf{x}_{(\cdot)} = \left[ ^>\mathbf{x}_{(1)}, ^>\mathbf{x}_{(2)}, ^>\mathbf{x}_{(3)}, ^>\mathbf{x}_{(4)} \right\rangle \in ^>\left(^>\mathbb{R}^k\right)^n$ и матрица $\mathbf{D} \in \operatorname{L}(\mathbb{R};m\cdot n)$ имеет матрично трёхдиагональный вид:

\begin{equation}\label{2}
	\mathbf{D}=\left(\begin{array}{llll}
	\mathbf{B} & \mathbf{C} & \mathbf{O} & \mathbf{O} \\
	\mathbf{A} & \mathbf{B} & \mathbf{C} & \mathbf{O} \\
	\mathbf{O} & \mathbf{A} & \mathbf{B} & \mathbf{C} \\
	\mathbf{O} & \mathbf{O} & \mathbf{A} & \mathbf{B}
	\end{array}\right),
\end{equation}

если

\begin{equation*}
	\mathbf{A}, \mathbf{B}, \mathbf{C} \in \operatorname{L}(\mathbb{R},3) \quad \textup{и} \quad \mathbf{O} \in \operatorname{L}(\mathbb{R},3) \textup{-- нулевая матрица}
\end{equation*}


\begin{equation*}
	A=(1-(k-63))\left(\begin{array}{ccc}
	N+2 & 1 & 0 \\
	1 & N+2 & 1 \\
	0 & 1 & N+2
	\end{array}\right), B=(64-k)\left(\begin{array}{lll}
	4 & 2 & 0 \\
	2 & 4 & 1 \\
	0 & 1 & 4
	\end{array} \right),
\end{equation*}

\begin{equation*}
	C=(-1)^{k}\left(\begin{array}{lll}
	3 & 1 & 0 \\
	1 & 3 & 1 \\
	0 & 1 & 3
	\end{array}\right)
\end{equation*}

\begin{equation*}
	 ^>\mathbf{f}_{(1)}= ^>\mathbf{f}_{(2)}= ^>\mathbf{f}_{(3)}= ^>\mathbf{f}_{(4)}=[N, N-k, N+k\rangle \in ^>\mathbb{R}^{3}
\end{equation*}



\pagebreak

\section*{Решение}

Начальные значения по условию: $k = 63, \quad N = 3$. Подставляя данные $k$ и $N$ имеем:

\begin{equation*}
	A=\left(\begin{array}{lll}
	9 & 1 & 0 \\
	1 & 9 & 1 \\
	0 & 1 & 9
	\end{array}\right), B=\left(\begin{array}{lll}
	4 & 1 & 0 \\
	1 & 4 & 1 \\
	0 & 1 & 4
	\end{array}\right), C=(-1)\left(\begin{array}{lll}
	3 & 1 & 0 \\
	1 & 3 & 1 \\
	0 & 1 & 3
	\end{array}\right) = 
	\left(\begin{array}{lll}
	-3 & -1 & 0 \\
	-1 & -3 & -1 \\
	0 & -1 & -3
	\end{array}\right)
\end{equation*}

\begin{equation*}
^>\mathbf{f}_{(1)}= ^>\mathbf{f}_{(2)}= ^>\mathbf{f}_{(3)}= ^>\mathbf{f}_{(4)}=[3, -60, 66\rangle \in ^>\mathbb{R}^{3}
\end{equation*}

Рассмотрим прямой ход метода матричной прогонки:

\begin{equation}\label{3}
	^>x_{k}=L_{k} ^{>}x_{k+1}+^{>}M_{k}, k=\overline{1, n-1}, \quad L_{k} \in L(\mathbb{R}, m), \quad>M_{k} \in^{>} \mathbb{R}^{m}
\end{equation}


\begin{equation*}
	^>x_{n}= ^{>}M_{n}, ^{>}M_{n} \in \mathbb{R}^{m}
\end{equation*}


\begin{equation*}
	L_{1}=-B_{1}^{-1} \cdot C_{1},^{>} M_{1}=B_{1}^{-1} \cdot^{>} f_{1}
\end{equation*}

\begin{equation*}
	L_{k}=-\left(L_{k-1} A_{k}+B_{k}\right)^{-1} \cdot C_{k},^{>} M_{k}=\left(L_{k-1} A_{k}+B_{k}\right)^{-1} \cdot\left(^{>} f_{k}-A_{k}^{>} M_{k-1}\right), k=\overline{2, n-1}
\end{equation*}

\begin{equation*}
	^>M_{n}=\left(L_{n-1} A_{n}+B_{n}\right)^{-1} \cdot\left(^>f_{n}-^{>} M_{n-1} A_{n}\right)=^>x_{n}
\end{equation*}


Будем выполнять работу в языке программирования R.

Введём начальные данные:

\begin{minted}{R}
> N <- 3
> N
[1] 3
> k <- 63
> k
[1] 63
\end{minted}

\begin{minted}{R}
> f <- matrix(c(N,N-k,N+k), nrow = 3, ncol = 1, byrow = TRUE)
> f 
     [,1]
[1,]    3
[2,]  -60
[3,]   66
\end{minted}

\pagebreak
Введём $\mathbf{A}, \mathbf{B}, \mathbf{C} \in \operatorname{L}(\mathbb{R},3)$:

\begin{minted}{R}
> A <- (1 - (k - 63)) *
+   matrix(c(N + 2, 1, 0, 1, N + 2, 1, 0, 1, N + 2), 
+          nrow = 3, ncol = 3, byrow = TRUE)
> A
     [,1] [,2] [,3]
[1,]    5    1    0
[2,]    1    5    1
[3,]    0    1    5
\end{minted}

\begin{minted}{R}
> B <- (64 - k) * 
+   matrix(c(4, 1, 0, 1, 4, 1, 0, 1, 4),
+          nrow = 3, ncol = 3, byrow = TRUE)
> B
     [,1] [,2] [,3]
[1,]    4    1    0
[2,]    1    4    1
[3,]    0    1    4
\end{minted}

\begin{minted}{R}
> C <- (-1)^k * 
+   matrix(c(3, 1, 0, 1, 3, 1, 0, 1, 3),
+          nrow = 3, ncol = 3, byrow = TRUE)
> C
     [,1] [,2] [,3]
[1,]   -3   -1    0
[2,]   -1   -3   -1
[3,]    0   -1   -3
\end{minted}

Тогда получим $L_{1}, L_{2}, L_{3}, M_{1}, M_{2}, M_{3}, M_4$:

\begin{minted}{R}
> L1 <- solve(-B) %*% C
> L1
     [,1]       [,2]        [,3]
[1,]  0.73214286 0.07142857 -0.01785714
[2,]  0.07142857 0.71428571  0.07142857
[3,] -0.01785714 0.07142857  0.73214286
> M1<-solve(B) %*% f
> M1
     [,1]
[1,]   6.267857
[2,] -22.071429
[3,]  22.017857
\end{minted}


\begin{minted}{R}
> L2 <- -(solve(L1 %*% A + B)) %*% C
> L2
     [,1]       [,2]         [,3]
[1,]  0.380084151 0.02945302 -0.007012623
[2,]  0.029453015 0.37307153  0.029453015
[3,] -0.007012623 0.02945302  0.380084151
\end{minted}


\begin{minted}{R}
> M2 <- (solve(L1 %*% A + B)) %*% (f - A %*% M1)
> M2
     [,1]
[1,] -2.023843
[2,]  4.493689
[3,] -4.056101
\end{minted}



\begin{minted}{R}
> L3 <- -(solve(L2 %*% A + B)) %*% C
> L3
     [,1]      [,2]        [,3]
[1,]  0.49443315 0.0447349 -0.01100163
[2,]  0.04473490 0.4834315  0.04473490
[3,] -0.01100163 0.0447349  0.49443315
\end{minted}


\begin{minted}{R}
> M3 <- (solve(L2 %*% A + B)) %*% (f - A %*% M2)
> M3
     [,1]
[1,]   6.436183
[2,] -19.360825
[3,]  18.762270
\end{minted}


\begin{minted}{R}
> M4 <- (solve(L3 %*% A + B)) %*% (f - A %*% M3)
> M4
     [,1]
[1,] -2.275742
[2,]  2.922765
[3,] -2.065917
\end{minted}

\pagebreak

Теперь рассмотрим обратный ход метода прогонки. Согласно прямому ходу метода прогонки, в равенствах (\ref{3}) для $k = \overline{1, n-1}$ вычислены все коэффициента $L_k$ и $^>M_k$ и последняя компонента $^>x_n = ^>M_n$ вектора неизвестных СЛАУ (\ref{1}) с матрицей вида (\ref{2}). Эти найденные матрицы и векторы позволяются последовательно найти все остальные компоненты вектора неизвестных СЛАУ (\ref{1}) с матрицей вида (\ref{2}):

\begin{equation*}
	\begin{aligned}
	&^>x_{n-1}=L_{n-1}\; ^{>}x_{n}+ ^{>}M_{n-1};\\
	&^>x_{n-2}=L_{n-2}\; ^{>}x_{n-2}+ ^{>}M_{n-2};\\
	& \ldots\ldots\ldots\ldots\ldots\ldots\ldots;\\
	&^>x_{1}=L_{1}\; ^{>}x_{2}+ ^{>}M_{1}.
	\end{aligned}
\end{equation*}

Тогда имеем:

\begin{minted}{R}
> x4 <- M4
> x4
     [,1]
[1,] -2.275742
[2,]  2.922765
[3,] -2.065917
> x3 <- L3 %*% x4 + M3
> x3
     [,1]
[1,]   5.464459
[2,] -18.142092
[3,]  17.896599
> x2 <- L2 %*% x3 + M2
> x2
     [,1]
[1,] -0.606730
[2,] -1.586556
[3,]  2.173453
> x1 <- L1 %*% x2 + M1
> x1
     [,1]
[1,]   5.671507
[2,] -23.092774
[3,]  23.506644
\end{minted}


\pagebreak

Сделаем проверку, что $\mathbf{D} \cdot ^>\mathbf{x}_{(\cdot)} = ^>\mathbf{f}_{(\cdot)}$:

Умножаемые в левой части уравнения имеют вид:

\begin{minted}{R}
> D
     [,1] [,2] [,3] [,4] [,5] [,6] [,7] [,8] [,9] [,10] [,11] [,12]
[1,]    4    1    0   -3   -1    0    0    0    0     0     0     0
[2,]    1    4    1   -1   -3   -1    0    0    0     0     0     0
[3,]    0    1    4    0   -1   -3    0    0    0     0     0     0
[4,]    5    1    0    4    1    0   -3   -1    0     0     0     0
[5,]    1    5    1    1    4    1   -1   -3   -1     0     0     0
[6,]    0    1    5    0    1    4    0   -1   -3     0     0     0
[7,]    0    0    0    5    1    0    4    1    0    -3    -1     0
[8,]    0    0    0    1    5    1    1    4    1    -1    -3    -1
[9,]    0    0    0    0    1    5    0    1    4     0    -1    -3
[10,]    0    0    0    0    0    0    5    1    0     4     1     0
[11,]    0    0    0    0    0    0    1    5    1     1     4     1
[12,]    0    0    0    0    0    0    0    1    5     0     1     4
\end{minted}

\begin{minted}{R}
> x
     [,1]
[1,]   5.671507
[2,] -23.092774
[3,]  23.506644
[4,]  -0.606730
[5,]  -1.586556
[6,]   2.173453
[7,]   5.464459
[8,] -18.142092
[9,]  17.896599
[10,]  -2.275742
[11,]   2.922765
[12,]  -2.065917
\end{minted}

\pagebreak
Тогда:

\begin{minted}{R}
> D%*%x
     [,1]
[1,]    3
[2,]  -60
[3,]   66
[4,]    3
[5,]  -60
[6,]   66
[7,]    3
[8,]  -60
[9,]   66
[10,]    3
[11,]  -60
[12,]   66
\end{minted}


Что в точности равняется $^>\mathbf{f}_{(\cdot)}$:

\begin{minted}{R}
> f1 < -matrix(c(f, f, f, f), nrow=12, ncol=1, byrow=TRUE)
> f1
     [,1]
[1,]    3
[2,]  -60
[3,]   66
[4,]    3
[5,]  -60
[6,]   66
[7,]    3
[8,]  -60
[9,]   66
[10,]    3
[11,]  -60
[12,]   66
\end{minted}

Видим, что проверка сошлась.












\end{document}