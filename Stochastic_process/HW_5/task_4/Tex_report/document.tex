\documentclass[12 pt]{article}  
\usepackage{cmap}
\usepackage[utf8]{inputenc}
\usepackage[T1,T2A]{fontenc}
\usepackage[english,russian]{babel}      	%sets the font to 12 pt and says this is an article (as opposed to book or other documents)
\usepackage{amssymb}				
\usepackage{mathtools}	% packages to get the fonts, symbols used in most math

%\usepackage{setspace}               		% Together with \doublespacing below allows for doublespacing of the document

\renewcommand{\thesubsection}{\arabic{subsection}}

\newenvironment{sqcases}{%
	\matrix@check\sqcases\env@sqcases
}{%
	\endarray\right.%
}
\def\env@sqcases{%
	\let\@ifnextchar\new@ifnextchar
	\left\lbrack
	\def\arraystretch{1.2}%
	\array{@{}l@{\quad}l@{}}%
}
\makeatother


\oddsidemargin=-0.5cm                 	% These three commands create the margins required for class
\setlength{\textwidth}{6.5in}         	%
\addtolength{\voffset}{-20pt}        		%
\addtolength{\headsep}{25pt}           	%

\newcommand*{\QEDB}{\hfill\ensuremath{\blacksquare}}%
\newcommand*{\QEDA}{\hfill\ensuremath{\square}}%


\pagestyle{myheadings}                           	% tells LaTeX to allow you to enter information in the heading
\markright{Sokolov Arseny\hfill \today \hfill} 	% put your name instead of Murphy Waggoner 
% and put the proposition number from the book
% LaTeX will put your name on the left, the date the paper 
% is generated in the middle 
% and a page number on the right



\newcommand{\eqn}[0]{\begin{array}{rcl}}%begin an aligned equation - allows for aligning = or inequalities.  Always use with $$ $$
	\newcommand{\eqnend}[0]{\end{array} }  	%end the aligned equation

\newcommand{\qed}[0]{$\square$}        	% make an unfilled square the default for ending a proof

%\doublespacing                         	% Together with the package setspace above allows for doublespacing of the document

\begin{document}												% end of preamble and beginning of text that will be printed
\begin{titlepage}
	\begin{center}
		\large
		МИНИСТЕРСТВО НАУКИ И ВЫСШЕГО ОБРАЗОВАНИЯ\\ РОССИЙСКОЙ ФЕДЕРАЦИИ
		
		\vspace{0.5cm}
		
		МГТУ им Н.Э.Баумана
		\vspace{0.25cm}
		
		Факультет ФН
		
		Кафедра вычислительной математики и математической физики
		\vfill
		
		
		Соколов Арсений Андреевич\\
		\vfill
		
		
		{\LARGE Домашнее задание №5 (задача 4) \\ по теории случайных процессов\\[2mm]
		}
		\bigskip
		
		3 курс, группа ФН11-63Б\\
		Вариант 19
	\end{center}
	\vfill
	
	\newlength{\ML}
	\settowidth{\ML}{«\underline{\hspace{0.7cm}}» \underline{\hspace{2cm}}}
	\hfill\begin{minipage}{0.4\textwidth}
		Преподаватель\\
		\underline{\hspace{3cm}} Т.\,В.~Облакова\\
		«\underline{\hspace{0.7cm}}» \underline{\hspace{1.71cm}} 2020 г.
	\end{minipage}%
	\bigskip
	
	
	\vfill
	
	\begin{center}
		Москва, 2020 г.
	\end{center}
\end{titlepage}
	
\textbf{Условие.}\\
Задана ковариационная функция стационарного случайного процесса $X(t)$. Найдите:
\begin{enumerate}
	\item ковариационную функцию, дисперсию и нормированную	ковариационную функцию случайного процесса $Y(t) = X'(t)$,
	\item взаимную ковариационную функцию $R_{X X^{\prime}}$,
	\item ковариационную функцию, дисперсию и нормированную	ковариационную функцию случайного процесса $Z(t) = X(t) + X'(t)$.
\end{enumerate}
	
\begin{equation}
	K_{X}(\tau)=(1+|\sin 3 \tau|) e^{-2|\tau|}
\end{equation}

\newpage
\textbf{Решение.}

\subsection{}

\begin{equation}
	K_{X}\left(t_{1}, t_{2}\right)=K_{X}\left(t_{2}-t_{1}\right)=K_{X}(\tau)=(1+|\sin 3 \tau|) e^{-2|\tau|}
\end{equation}

\begin{equation}
	Y(t) = X'(t)
\end{equation}


\begin{equation}
	K_{Y}\left(t_{1}, t_{2}\right)=K_{Y}\left(t_{2}-t_{1}\right)=K_{Y}(\tau)
\end{equation}

Ковариационная функция:

\begin{equation}
	\begin{array}{l}
	K_{Y}(\tau)=K_{X^{\prime}}(\tau)=-K_{X}^{\prime \prime}(\tau)=-\left(3|\cos 3 \tau| e^{-2|\tau|}-2(1+|\sin 3 \tau|) e^{-2|\tau|}\right)^{\prime}= \\
	=-\left(e^{-2|\tau|}(3|\cos 3 \tau|-2-2|\sin 3 \tau|)\right)^{\prime}=-\left(-2 e^{-2|\tau|}(3|\cos 3 \tau|-2-2|\sin 3 \tau|)+\right. \\
	\left.+e^{-2|\tau|}(-9|\sin 3 \tau|-6|\cos 3 \tau|)\right)=-\left(e^{-2|\tau|}(-12|\cos 3 \tau|-5|\sin 3 \tau|+4)\right)= \\
	\left.+e^{-2|\tau|}(-9|\sin 3 \tau|-6|\cos 3 \tau|)\right)=e^{-2|\tau|}(12|\cos 3 \tau|+5|\sin 3 \tau|-4)
	\end{array}
\end{equation}

Дисперсия:

\begin{equation}
	D_{Y}(t)=K_{Y}(0)=e^{-2(0)}(12|\cos 0|+5|\sin 0|-4)=12-4=8
\end{equation}

\begin{equation*}
\Updownarrow
\end{equation*}

\begin{equation}
	D_{Y}(t)=K_{Y}(t, t)=e^{-2|t-t|}(12|\cos 3(t-t)|+5|\sin 3(t-t)|-4)=8
\end{equation}

Нормированная ковариационная функция:

\begin{equation}
	\rho_{Y}(\tau)=\frac{K_{Y}(\tau)}{K_{Y}(0)}=\frac{e^{-2|\tau|}(12|\cos 3 \tau|+5|\sin 3 \tau|-4)}{8}
\end{equation}



\newpage
\subsection{}
Взаимная ковариационная функция:

\begin{equation}
	R_{X X^{\prime}}(\tau)=K_{X}^{\prime}(\tau)=\left((1+|\sin 3 \tau|) e^{-2|\tau|}\right)^{\prime}=e^{-2|\tau|}(3|\cos 3 \tau|-2-2|\sin 3 \tau|)
\end{equation}


\newpage

\subsection{}

\begin{equation}
	Z(t) = X(t) + X'(t)
\end{equation}

Ковариационная функция:

\begin{equation}
	\begin{array}{l}
	K_{Z}(\tau)=\operatorname{cov}\left(X\left(t_{1}\right)+X^{\prime}\left(t_{1}\right), X\left(t_{2}\right)+X^{\prime}\left(t_{2}\right)\right)=K_{X}(\tau)+R_{X X^{\prime}}(\tau)+R_{X^{\prime} X}(\tau)+K_{X^{\prime}}(\tau)= \\
	=K_{X}(\tau)+K_{X}^{\prime}(\tau)-K_{X}^{\prime}(\tau)-K_{X}^{\prime \prime}(\tau)=K_{X}(\tau)-K_{X}^{\prime \prime}(\tau)=(1+|\sin 3 \tau|) e^{-2|\tau|}+ \\
	+e^{-2|\tau|}(12|\cos 3 \tau|+5|\sin 3 \tau|-4)=e^{-2|\tau|}(-3+6|\sin 3 \tau|+12|\cos 3 \tau|)
	\end{array}
\end{equation}

Дисперсия:

\begin{equation}
	D_{Z}(t)=K_{Z}(0)==e^{-2(0)}(-3+6|\sin 0|+12|\cos 0|)=-3+12=9
\end{equation}

Нормированная ковариационная функция:

\begin{equation}
	\rho_{\mathrm{Z}}(\tau)=\frac{K_{\mathrm{Z}}(\tau)}{K_{\mathrm{Z}}(0)}=\frac{e^{-2|\tau|}(-3+6|\sin 3 \tau|+12|\cos 3 \tau|)}{9}=\frac{e^{-2|\tau|}(-3+2|\sin 3 \tau|+4|\cos 3 \tau|)}{3}
\end{equation}






















\end{document}