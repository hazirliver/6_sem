\documentclass[12 pt]{article}  
\usepackage{cmap}
\usepackage[utf8]{inputenc}
\usepackage[T1,T2A]{fontenc}
\usepackage[english,russian]{babel}      	%sets the font to 12 pt and says this is an article (as opposed to book or other documents)
\usepackage{amssymb}				
\usepackage{mathtools}	% packages to get the fonts, symbols used in most math

%\usepackage{setspace}               		% Together with \doublespacing below allows for doublespacing of the document

\renewcommand{\thesubsection}{\arabic{subsection}}

\newenvironment{sqcases}{%
	\matrix@check\sqcases\env@sqcases
}{%
	\endarray\right.%
}
\def\env@sqcases{%
	\let\@ifnextchar\new@ifnextchar
	\left\lbrack
	\def\arraystretch{1.2}%
	\array{@{}l@{\quad}l@{}}%
}
\makeatother


\oddsidemargin=-0.5cm                 	% These three commands create the margins required for class
\setlength{\textwidth}{6.5in}         	%
\addtolength{\voffset}{-20pt}        		%
\addtolength{\headsep}{25pt}           	%

\newcommand*{\QEDB}{\hfill\ensuremath{\blacksquare}}%
\newcommand*{\QEDA}{\hfill\ensuremath{\square}}%


\pagestyle{myheadings}                           	% tells LaTeX to allow you to enter information in the heading
\markright{Sokolov Arseny\hfill \today \hfill} 	% put your name instead of Murphy Waggoner 
% and put the proposition number from the book
% LaTeX will put your name on the left, the date the paper 
% is generated in the middle 
% and a page number on the right



\newcommand{\eqn}[0]{\begin{array}{rcl}}%begin an aligned equation - allows for aligning = or inequalities.  Always use with $$ $$
	\newcommand{\eqnend}[0]{\end{array} }  	%end the aligned equation

\newcommand{\qed}[0]{$\square$}        	% make an unfilled square the default for ending a proof

%\doublespacing                         	% Together with the package setspace above allows for doublespacing of the document

\begin{document}												% end of preamble and beginning of text that will be printed
\begin{titlepage}
	\begin{center}
		\large
		МИНИСТЕРСТВО НАУКИ И ВЫСШЕГО ОБРАЗОВАНИЯ\\ РОССИЙСКОЙ ФЕДЕРАЦИИ
		
		\vspace{0.5cm}
		
		МГТУ им Н.Э.Баумана
		\vspace{0.25cm}
		
		Факультет ФН
		
		Кафедра вычислительной математики и математической физики
		\vfill
		
		
		Соколов Арсений Андреевич\\
		\vfill
		
		
		{\LARGE Домашнее задание №5 (задача 3) \\ по теории случайных процессов\\[2mm]
		}
		\bigskip
		
		3 курс, группа ФН11-63Б\\
		Вариант 19
	\end{center}
	\vfill
	
	\newlength{\ML}
	\settowidth{\ML}{«\underline{\hspace{0.7cm}}» \underline{\hspace{2cm}}}
	\hfill\begin{minipage}{0.4\textwidth}
		Преподаватель\\
		\underline{\hspace{3cm}} Т.\,В.~Облакова\\
		«\underline{\hspace{0.7cm}}» \underline{\hspace{1.71cm}} 2020 г.
	\end{minipage}%
	\bigskip
	
	
	\vfill
	
	\begin{center}
		Москва, 2020 г.
	\end{center}
\end{titlepage}
	
\textbf{Условие.}\\
Задан случайный процесс $X(t)$. Найдите (не дифференцируя и не интегрируя $X(t)$):
\begin{enumerate}
	\item Математическое ожидание $m_X(t) = M[X(t)]$, ковариационную функцию $K_X(t_1,T_2)$ и дисперсию $D_X(t)$ случайного процесса $X(t)$;
	\item Математическое ожидание, ковариационную функцию и дисперсию случайного процесса $T_1(t) = \frac{dX(t)}{dt}$;
	\item Математическое ожидание, ковариационную функцию и дисперсию случайного процесса $Y_2(t) = X(t) + \frac{dX(t)}{dt}$;
	\item Математическое ожидание, ковариационную функцию и дисперсию случайного процесса $Y_3(t) = \int_{0}^{t}X(s)ds$;
	\item Взаимные ковариационные функции $R_{XX'}(t_1,t_2)$ и $R_{X'X}(t_1,t_2)$.
\end{enumerate}
	
\begin{equation}
	X(t) = U\cos(t) + Vt^3,
\end{equation}
	где $U$ и $V$ -- некоррелированные случайные величины, $MU = MV = 0, \; {DU = 2, DV = 1}$.
\newpage
\textbf{Решение.}

\subsection{}

Математическое ожидание:

\begin{equation}
	m_{X}(t)=M[X(t)]=M U \cos t+M V t^{3}=0
\end{equation}

Ковариационная функция:

\begin{equation}
	\begin{aligned}
	&K_{X}\left(t_{1}, t_{2}\right)=\operatorname{cov}\left(U \cos t_{1}+V t_{1}^{3}, \quad U \cos t_{2}+V t_{2}^{3}\right)=\cos t_{1} \cos t_{2} \operatorname{cov}(U, U)+\\
	&+t_{2}^{3} \cos t_{1} \operatorname{cov}(U, V)+t_{1}^{3} \cos t_{2} \operatorname{cov}(V, U)+t_{1}^{3} t_{2}^{3} \operatorname{cov}(V, V)
	\end{aligned}
\end{equation}


Так как $U$ и $V$ -- некоррелированные случайные величины, следовательно, 

\begin{equation}
	\operatorname{cov}(U, V) = \operatorname{cov}(V, U) = 0
\end{equation}


Получаем:

\begin{equation}
	K_{X}\left(t_{1}, t_{2}\right)=2 \cos t_{1} \cos t_{2}+t_{1}^{3} t_{2}^{3}
\end{equation}

Дисперсия:

\begin{equation}
	D_{X}(t)=K_{X}(t, t)=2 \operatorname{cos}t \cos t+t^{3} t^{3}=2 \cos ^{2} t+t^{6}
\end{equation}

\newpage
\subsection{}

\begin{equation}
	Y_1(t) = \frac{dX(t)}{dt}
\end{equation}

Математическое ожидание:

\begin{equation}
	m_{Y_{1}}(t)=\frac{d}{d t} M[X(t)]=\frac{d}{d t} 0=0
\end{equation}


Ковариационная функция:

\begin{align*}
	&K_{Y_{1}}\left(t_{1}, t_{2}\right)=\frac{\partial^{2}}{\partial t_{1} \partial t_{2}} K_{X}\left(t_{1}, t_{2}\right)=\frac{\partial^{2}}{\partial t_{1} \partial t_{2}}\left(2 \cos t_{1} \cos t_{2}+t_{1}^{3} t_{2}^{3}\right)=\frac{\partial}{\partial t_{2}}\left(-2 \sin t_{1} \cos t_{2}+3 t_{1}^{2} t_{2}^{3}\right)= \\
	&=2 \sin t_{1} \sin t_{2}+9 t_{1}^{2} t_{2}^{2}
\end{align*}


Дисперсия:

\begin{equation}
	D_{Y_{1}}(t)=K_{Y_{1}}(t, t)=2 \operatorname{sin}t\operatorname{sin}t+9 t^{2} t^{2}=2 \sin ^{2} t+9 t^{4}
\end{equation}


\newpage
\subsection{}

\begin{equation}
	Y_2(t) = X(t) + \frac{dX(t)}{dt}
\end{equation}

Математическое ожидание:

\begin{equation}
	m_{Y_{2}}(t)=\frac{d}{d t} M\left[X(t)+X^{\prime}(t)\right]=M[X(t)]+M\left[X^{\prime}(t)\right]=m_{X}(t)+\left(m_{X}(t)\right)^{\prime}=0
\end{equation}


Ковариационная функция:

\begin{equation}
	\begin{array}{l}
	K_{Y_{2}}\left(t_{1}, t_{2}\right)=\operatorname{cov}\left(X\left(t_{1}\right)+X^{\prime}\left(t_{1}\right), X\left(t_{2}\right)+X^{\prime}\left(t_{2}\right)\right)=K_{X}\left(t_{1}, t_{2}\right)+R_{X X^{\prime}}\left(t_{1}, t_{2}\right)+ \\
	+R_{X^{\prime} X}\left(t_{1}, t_{2}\right)+K_{X^{\prime}}\left(t_{1}, t_{2}\right)=K_{X}\left(t_{1}, t_{2}\right)+\frac{\partial}{\partial t_{1}} K_{X}\left(t_{1}, t_{2}\right)+\frac{\partial}{\partial t_{2}} K_{X}\left(t_{1}, t_{2}\right)+\frac{\partial^{2}}{\partial t_{1} \partial t_{2}} K_{X}\left(t_{1}, t_{2}\right)= \\
	=2 \cos t_{1} \cos t_{2}+t_{1}^{3} t_{2}^{3}-2 \sin t_{1} \cos t_{2}+3 t_{1}^{2} t_{2}^{3}-2 \cos t_{1} \sin t_{2}+3 t_{1}^{3} t_{2}^{2}+2 \sin t_{1} \sin t_{2}+9 t_{1}^{2} t_{2}^{2}= \\
	=2 \cos \left(t_{1}-t_{2}\right)-2 \sin \left(t_{1}+t_{2}\right)+t_{1}^{3} t_{2}^{3}+3 t_{1}^{2} t_{2}^{3}+3 t_{1}^{3} t_{2}^{2}+9 t_{1}^{2} t_{2}^{2}
	\end{array}
\end{equation}

Дисперсия:

\begin{equation}
	\begin{array}{l}
	D_{Y_{2}}(t)=K_{Y_{2}}(t, t)=2 \cos (t-t)-2 \sin (t+t)+t^{3} t^{3}+3 t^{2} t^{3}+3 t^{3} t^{2}+9 t^{2} t^{2}=2 \cos (0)- \\
	-2 \sin 2 t+t^{6}+3 t^{5}+3 t^{5}+9 t^{4}=2-2 \sin 2 t+t^{6}+6 t^{5}+9 t^{4}=2-2 \sin 2 t+\left(t^{3}+3 t^{2}\right)^{2}
	\end{array}
\end{equation}

\newpage
\subsection{}

\begin{equation}
	Y_3(t) = \int_{0}^{t}X(s)ds
\end{equation}


Математическое ожидание:

\begin{equation}
	m_{Y_{3}}(t)=M\left[Y_{3}(t)\right]=M\left[\int_{0}^{t} X(s) d s\right]=\int_{0}^{t} m_{X}(s) d s=\int_{0}^{t} 0 d s=0
\end{equation}


Ковариационная функция:

\begin{equation}
	\begin{aligned}
	&K_{Y_{3}}\left(t_{1}, t_{2}\right)=\int_{0}^{t_{1} t_{2}} \int_{0}^{t_{1}} K_{X}\left(s_{1}, s_{2}\right) d s_{1} d s_{2}=\int_{0}^{t_{1} t_{2}}\left(2 \cos t_{1} \cos t_{2}+t_{1}^{3} t_{2}^{3}\right) d s_{1} d s_{2}=\\
	&=\frac{1}{16} t_{1}^{4} t_{2}^{4}+2 \sin t_{1} \sin t_{2}
	\end{aligned}
\end{equation}


Дисперсия:

\begin{equation}
	D_{Y_{3}}(t)=K_{Y_{3}}(t, t)=\frac{1}{16} t^{4} t^{4}+2 \sin t \sin t=\frac{1}{16} t^{8}+2 \sin ^{2} t
\end{equation}

\newpage
\subsection{}

\begin{equation}
	\begin{array}{c}
	R_{X X^{\prime}}\left(t_{1}, t_{2}\right)=\frac{\partial K_{X}\left(t_{1}, t_{2}\right)}{\partial t_{2}}=\frac{\partial}{\partial t_{2}}\left(2 \cos t_{1} \cos t_{2}+t_{1}^{3} t_{2}^{3}\right)=-2 \cos t_{1} \sin t_{2}+3 t_{1}^{3} t_{2}^{2} \\
	R_{X^{\prime} X}\left(t_{1}, t_{2}\right)=\frac{\partial K_{X}\left(t_{1}, t_{2}\right)}{\partial t_{1}}=\frac{\partial}{\partial t_{1}}\left(2 \cos t_{1} \cos t_{2}+t_{1}^{3} t_{2}^{3}\right)=-2 \sin t_{1} \cos t_{2}+3 t_{1}^{2} t_{2}^{3}
	\end{array}
\end{equation}















\end{document}