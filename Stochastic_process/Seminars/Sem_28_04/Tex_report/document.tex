\documentclass[12 pt]{article}  
\usepackage{cmap}
\usepackage[utf8]{inputenc}
\usepackage[T1,T2A]{fontenc}
\usepackage[english,russian]{babel}      	%sets the font to 12 pt and says this is an article (as opposed to book or other documents)
\usepackage{amssymb}				
\usepackage{mathtools}	% packages to get the fonts, symbols used in most math

%\usepackage{setspace}               		% Together with \doublespacing below allows for doublespacing of the document

\oddsidemargin=-0.5cm                 	% These three commands create the margins required for class
\setlength{\textwidth}{6.5in}         	%
\addtolength{\voffset}{-20pt}        		%
\addtolength{\headsep}{25pt}           	%

\newcommand*{\QEDB}{\hfill\ensuremath{\blacksquare}}%
\newcommand*{\QEDA}{\hfill\ensuremath{\square}}%


\pagestyle{myheadings}                           	% tells LaTeX to allow you to enter information in the heading
\markright{Sokolov Arseny\hfill \today \hfill} 	% put your name instead of Murphy Waggoner 
% and put the proposition number from the book
% LaTeX will put your name on the left, the date the paper 
% is generated in the middle 
% and a page number on the right



\newcommand{\eqn}[0]{\begin{array}{rcl}}%begin an aligned equation - allows for aligning = or inequalities.  Always use with $$ $$
	\newcommand{\eqnend}[0]{\end{array} }  	%end the aligned equation

\newcommand{\qed}[0]{$\square$}        	% make an unfilled square the default for ending a proof

%\doublespacing                         	% Together with the package setspace above allows for doublespacing of the document

\begin{document}												% end of preamble and beginning of text that will be printed
	
	% makes the word Proposition and the proposition number bold face  
	\textbf{Proposition 1:}							% the Proposition number from the book (this one is fictitious)
	Доказать, что $\left|\sum\limits_{k=0}^{n-1} \varphi\left(s, t_{k}^{\prime}\right) K_{\xi}\left(t, t_{k}^{\prime}\right) \Delta t_{k}-R_{\xi \zeta}(t, s)\right| \rightarrow 0$
	\\
	% be sure to leave at least one blank line here so that 
	% the Proof starts with a new paragraph
	
	\textbf{Proof:}            						% makes the word Proof bold face
	\begin{align*}
		&\left|\sum\limits_{k=0}^{n-1} \varphi\left(s, t_{k}^{\prime}\right) K_{\xi}\left(t, t_{k}^{\prime}\right) \Delta t_{k}-R_{\xi \zeta}(t, s)\right|  = 
		\left| \sum\limits_{k=0}^{n-1} \varphi(s,t'_k)(\operatorname{M}\xi_t\xi_{t'_k} - \operatorname{M}\xi_t\operatorname{M}\xi_{t'_k})\Delta t_k - \operatorname{M}\xi_t\zeta_s + \operatorname{M}\xi_t\operatorname{M}\zeta_s\right| \leq \\
		& \leq \left| \sum\limits_{k=0}^{n-1} \varphi(s,t'_k)\operatorname{M}\xi_t\xi_{t'_k}\Delta t_k - \operatorname{M}\xi_t\zeta_s \right| + \left| \sum\limits_{k=0}^{n-1}\varphi(s,t'_k)\operatorname{M}\operatorname{\xi}\operatorname{M}\xi_{t'_k}\Delta t_k - \operatorname{M}\xi_t\operatorname{M}\zeta_s \right|
	\end{align*}
	
	Докажем, что данное выражение стремится к нулю. Для этого рассмотрим каждое слагаемое по отдельности. Для первого слагаемого имеем:
	
	\begin{align*}
		&\left| \sum\limits_{k=0}^{n-1} \varphi(s,t'_k)\operatorname{M}\xi_t\xi_{t'_k}\Delta t_k - \operatorname{M}\xi_t\zeta_s \right| = 
		\left| \operatorname{M} \left( \sum\limits_{k=0}^{n-1} \varphi(s,t'_k)\xi_t\xi_{t'_k}\Delta t_k - \xi_t\zeta_s \right) \right| = \\
		& = \left|  \operatorname{M} \left[ \xi_t \left( \sum\limits_{k=0}^{n-1}\varphi(s,t'_k)\xi_{t'_k} \Delta t_n - \zeta_s \right) \right] \right| \leq \operatorname{M} \left| \xi_t \left( \sum\limits_{k=0}^{n-1} \varphi(s,t'_k)\xi'_{t_k} \Delta t_k -\zeta_s \right) \right| \leq \\
		& \leq \sqrt{\operatorname{M} \xi_t^2} \cdot \sqrt{\operatorname{M} \left| \sum\limits_{k=0}^{n-1} \varphi(s,t'_k)\xi_{t'_k} \Delta t_k - \zeta_s \right|^2} \longrightarrow 0
	\end{align*}
	
	Для второго слагаемого:
	
	\begin{align*}
		&\left| \sum\limits_{k=0}^{n-1} \varphi(s,t'_k)\operatorname{M}\xi_t \operatorname{M}\xi_{t'_k}\Delta t_k - \operatorname{M}\xi_t \operatorname{M}\zeta_s \right| = \left| \operatorname{M}\operatorname{\xi_{t}} \left( \sum\limits_{k=0}^{n-1} \varphi(s,t'_k) \operatorname{M}\xi_{t'_k} \Delta t_k  - \operatorname{M}\zeta_s\right) \right| \leq \\
		& \leq \left|\operatorname{M\xi_t}\right| \cdot \left| \operatorname{M}\sum\limits_{k=0}^{n-1}\varphi(s,t'_k) \xi_{t'_k} \Delta t_k  - \zeta_s\right| \leq 
		\left|\operatorname{M\xi_t}\right| \cdot \operatorname{M} \sqrt{\left| \sum\limits_{k=0}^{n-1} \varphi(s,t'_k) \xi_{t'_k} \Delta t_k - \zeta_s  \right|^2} \longrightarrow 0
	\end{align*}
	
	
	Таким образом:
	
	\begin{align*}
		&\left|\sum\limits_{k=0}^{n-1} \varphi\left(s, t_{k}^{\prime}\right) K_{\xi}\left(t, t_{k}^{\prime}\right) \Delta t_{k}-R_{\xi \zeta}(t, s)\right| = \\
		 &= \left| \sum\limits_{k=0}^{n-1} \varphi(s,t'_k)\operatorname{M}\xi_t\xi_{t'_k}\Delta t_k - \operatorname{M}\xi_t\zeta_s \right| + \left| \sum\limits_{k=0}^{n-1}\varphi(s,t'_k)\operatorname{M}\operatorname{\xi}\operatorname{M}\xi_{t'_k}\Delta t_k - \operatorname{M}\xi_t\operatorname{M}\zeta_s \right|
		  \longrightarrow 0 + 0 = 0 
	\end{align*}
\end{document}