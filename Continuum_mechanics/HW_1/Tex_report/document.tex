\documentclass[14pt,a4paper]{scrartcl}
\usepackage{cmap}
\usepackage[utf8]{inputenc}
\usepackage[T1,T2A]{fontenc}
\usepackage[english,russian]{babel}
\usepackage{relsize}
\usepackage{graphicx}
\usepackage{subfigure}
\usepackage{mathtools}
\usepackage{amssymb}
\usepackage{float}
\usepackage{sidecap}
\usepackage{wrapfig}
\usepackage{caption}
\usepackage[table,xcdraw]{xcolor}
\usepackage{minted}
\usepackage{tcolorbox}
\usepackage{enumitem}
\usepackage{physics}
\makeatletter

\renewcommand{\thesubsection}{\arabic{subsection}}

\newenvironment{sqcases}{%
	\matrix@check\sqcases\env@sqcases
}{%
	\endarray\right.%
}
\def\env@sqcases{%
	\let\@ifnextchar\new@ifnextchar
	\left\lbrack
	\def\arraystretch{1.2}%
	\array{@{}l@{\quad}l@{}}%
}
\makeatother

\begin{document}
	\begin{titlepage}
	\begin{center}
		\large
		МИНИСТЕРСТВО НАУКИ И ВЫСШЕГО ОБРАЗОВАНИЯ\\ РОССИЙСКОЙ ФЕДЕРАЦИИ
		
		\vspace{0.5cm}
		
		МГТУ им Н.Э.Баумана
		\vspace{0.25cm}
		
		Факультет ФН
		
		Кафедра вычислительной математики и математической физики
		\vfill
		
		
		Соколов Арсений Андреевич\\
		\vfill
		
		
		{\LARGE Домашнее задания №1 \\ по механике сплошной среды\\[2mm]
		}
		\bigskip
		
		3 курс, группа ФН11-63Б\\
		Вариант 19
	\end{center}
	\vfill
	
	\newlength{\ML}
	\settowidth{\ML}{«\underline{\hspace{0.7cm}}» \underline{\hspace{2cm}}}
	\hfill\begin{minipage}{0.4\textwidth}
		Преподаватель\\
		\underline{\hspace{3cm}} Е.\,А.~Губарева\\
		«\underline{\hspace{0.7cm}}» \underline{\hspace{1.71cm}} 2020 г.
	\end{minipage}%
	\bigskip
	
	
	\vfill
	
	\begin{center}
		Москва, 2020 г.
	\end{center}
\end{titlepage}

\section*{Задание}
Рассмотрим сплошную среду $B$, которая в $\mathcal{\overset{\circ}{K}}$ представляет собой прямоугольный параллелепипед (брус), который при переходе в $\mathcal{K}$ изменяет свои линейные размеры без изменения углов и поворачивается на угол $\varphi(t)$ в плоскости $O\mathrm{x}^1\mathrm{x}^2$ вокруг точки $O$. Закон движения такого тела называют вращением бруса с растяжением. Соотношения для него имеют вид:

\begin{equation*}
	\mathrm{x}^{\mathrm{i}}=\mathrm{F}_{0 \mathrm{j}}^{\mathrm{i}} \mathrm{X}^{\mathrm{i}}, \quad \dot{\mathrm{x}}^{\mathrm{i}}=\mathrm{X}^{\mathrm{i}},
\end{equation*}
где матрица $F_{0j}^i$ представляет собой произведение двух матриц (матрица вращения $O_0$ и матрицы растяжения $U_0$):

\begin{align*}
	\begin{aligned}
	O_{0 j}^{i}&=\left(\begin{array}{ccc}
	\cos \varphi & -\sin \varphi & 0 \\
	\sin \varphi & \cos \varphi & 0 \\
	0 & 0 & 1
	\end{array}\right)^{T}, & U_{0}^{i}=\left(\begin{array}{ccc}
	k_{1} & 0 & 0 \\
	0 & k_{2} & 0 \\
	0 & 0 & k_{3}
	\end{array}\right),  \\
	& F_{0 j}^{i} = \left(\begin{array}{ccc}
	k_{1} \cos \varphi & -k_{1} \sin \varphi & 0 \\
	k_{2} \sin \varphi & k_{2} \cos \varphi & 0 \\
	0 & 0 & k_{3}
	\end{array}\right)^{T};
	\end{aligned}
\end{align*}

$k_\alpha(t)\frac{h_\alpha(t)}{h_\alpha^0}$ -- функции пропорциональности, характеризующие отношения длин бруса в $\mathcal{K}$ и $\mathcal{\overset{\circ}{K}}$.

Вводя тензоры поворота $O_0$ и растяжения $U_0$:

\begin{equation*}
	0_{0}=0_{0 j}^{i} \bar{e}_{i} \otimes \bar{e}^{i}, \quad U_{0}=\sum_{\alpha=1}^{3} k_{\alpha} \bar{e}_{\alpha} \otimes \bar{e}_{\alpha}
\end{equation*}

закон движения бруса можно записать в тензорной форме:
\begin{equation*}
	\mathbf{x}=\mathbf{F}_{0} \cdot \dot{\mathbf{x}}, \quad \mathbf{F}_{0}=\mathbf{U}_{0} \cdot \mathbf{0}_{0}
\end{equation*}

\textbf{Найти:}
\begin{enumerate}
	\item Локальные векторы базиса
	\item Метрические матрицы
	\item Градиент деформации
	\item Тензоры деформации Коши-Грина и Альманси
\end{enumerate}

\pagebreak

\section{Локальные векторы базиса}

Найдём локальные векторы базиса в отсчетной и актуальной конфигурациях. Учитывая, что матрица Якоби $Q_i^{\;j} = \pdv{x^j}{X^i}$ совпадает с матрицей $F_{0i}^j$, имеем:

\begin{equation*}
\overset{\circ}{\bar{R}_{i}}=\frac{\partial \overset{\circ}{\bar{R}_{i}}}{\partial X^{i}}=\frac{\partial \overset{\circ}{x^j}}{\partial X^{i}} \overline{e_{j}}= \frac{X^j}{X^i}\overline{e_{j}} = \delta_{i}^j\overline{e_{j}}=\overline{e_{i}}
\end{equation*}


\begin{equation*}
\bar{R}_{i}=\frac{\partial \bar{x}}{\partial X^{i}}=\frac{\partial x^{j}}{\partial X^{i}} \overline{e_{j}}=\frac{\partial\left(F_{0 k}^{j} X^{k}\right)}{\partial X^{i}} \overline{e_{j}}=\delta_{i}^{k} F_{0 k}^{j} \overline{e_{j}}=F_{0 i}^{j} \overline{e_{j}}
\end{equation*}

\begin{equation*}
	\bar{R}_{1}=k_{1} \cos \varphi \bar{e}_{1}-k_{1} \sin \varphi \bar{e}_{2}
\end{equation*}


\begin{equation*}
	\bar{R}_{2}=k_{2} \sin \varphi \bar{e}_{1}+k_{2} \cos \varphi \bar{e}_{2}
\end{equation*}

\begin{equation*}
	\bar{R}_{3}=k_{3} \bar{e}_{3}
\end{equation*}



\section{Метрические матрицы}




\begin{equation*}
	\overset{\circ}{g}_{ij} = \overset{\circ}{\overline{R}}_i \cdot \overset{\circ}{\overline{R}}_j = \delta_{ij}
\end{equation*}


\begin{equation*}
	g_{i j}=\bar{R}_{i} \cdot \bar{R}_{j}=F_{0 i}^{k} \overline{e_{k}} \cdot F_{0 j}^{l} \overline{e_{l}}=F_{0 i}^{k} F_{0 j}^{l} \delta_{k l}	
\end{equation*}


\begin{equation*}
	g_{11}=\bar{R}_{1} \cdot \bar{R}_{1}=k_{1}^{2} \cos ^{2} \varphi+k_{1}^{2} \sin ^{2} \varphi=k_{1}^{2}
\end{equation*}

\begin{equation*}
	g_{12}=g_{21}=\bar{R}_{1} \cdot \bar{R}_{2}=k_{1} k_{2} \cos \varphi \sin \varphi-k_{1} k_{2} \cos \varphi \sin \varphi=0
\end{equation*}

\begin{equation*}
	g_{13}=g_{31}=\bar{R}_{1} \cdot \bar{R}_{3}=0
\end{equation*}

\begin{equation*}
	g_{23}=g_{32}=\bar{R}_{2} \cdot \bar{R}_{3}=0
\end{equation*}

\begin{equation*}
	g_{33}=k_{3}^{2}
\end{equation*}

То есть метрическая матрица имеет вид:

\begin{equation*}
	\left(g_{i j}\right)=\left(\begin{array}{ccc}
	k_{1}^{2} & 0 & 0 \\
	0 & k_{2}^{2} & 0 \\
	0 & 0 & k_{3}^{2}
	\end{array}\right)
\end{equation*}

А обратная метрическая матрица:

\begin{equation*}
	\left(g^{i j}\right)=\left(\begin{array}{ccc}
	\frac{1}{k_{1}^{2}} & 0 & 0 \\
	0 & \frac{1}{k_{2}^{2}} & 0 \\
	0 & 0 & \frac{1}{k_{3}^{2}}
	\end{array}\right)
\end{equation*}



\section{Градиент деформации}

Для расчёта градиента деформации найдём векторы взаимного базиса в отсчетной

\begin{equation*}
	\overset{\circ}{\overline{R^i}} = \overset{\circ}{g^{ij}} \overset{\circ}{\overline{R}}_j = \overline{e_{i}}
\end{equation*}

и актуальной конфигурации:


\begin{equation*}
	\bar{R}^{i}=g^{i j} \bar{R}_{j}
\end{equation*}


\begin{equation*}
	\bar{R}^{1}=g^{1 j} \bar{R}_{j}=g^{11} \bar{R}_{1}=\frac{1}{k_{1}} \cos \varphi \bar{e}_{1}-\frac{1}{k_{1}} \sin \varphi \bar{e}_{2}
\end{equation*}


\begin{equation*}
	\bar{R}^{2}=g^{2 j} \bar{R}_{j}=g^{22} \bar{R}_{2}=\frac{1}{k_{2}} \sin \varphi \bar{e}_{1}+\frac{1}{k_{2}} \cos \varphi \bar{e}_{2}
\end{equation*}


\begin{equation*}
	\bar{R}^{3}=g^{3 j} \bar{R}_{j}=g^{33} \bar{R}_{3}=\frac{1}{k_{3}} \bar{e}_{3}
\end{equation*}



Тогда градиент деформации имеет вид:

\begin{align*}
	\begin{aligned}
	&F=\bar{R}_{i} \otimes \overset{\circ}{\overline{R^i}}=\bar{R}_{1} \otimes \overset{\circ}{\overline{R^1}}+\bar{R}_{2} \otimes \overset{\circ}{\overline{R^2}}+\bar{R}_{3} \otimes \overset{\circ}{\overline{R^3}}=\\
	&=\left(k_{1} \cos \varphi \overline{e_{1}}-k_{1} \sin \varphi \overline{e_{2}}\right) \otimes \bar{e}_{1}+\left(k_{2} \sin \varphi \overline{e_{1}}+k_{2} \cos \varphi \overline{e_{2}}\right) \otimes \bar{e}_{2}+k_{3} \bar{e}_{3} \otimes \bar{e}_{3}=\\
	&=k_{1} \cos \varphi\bar{e_{1}} \otimes \bar{e}_{1}+k_{2} \cos \varphi \bar{e_{2}} \otimes \bar{e}_{2}+k_{3} \bar{e}_{3} \otimes \bar{e}_{3}-k_{1} \sin \varphi \bar{e}_{2} \otimes \bar{e}_{1}+k_{2} \sin \varphi \bar{e}_{1} \otimes \bar{e}_{2}
	\end{aligned}
\end{align*}


В матричном виде:

\begin{equation*}
	\left(F^{i j}\right)=\left(\begin{array}{ccc}
	k_{1} \cos \varphi & k_{2} \sin \varphi & 0 \\
	-k_{1} \sin \varphi & k_{2} \cos \varphi & 0 \\
	0 & 0 & k_{3}
	\end{array}\right); \quad \quad \det(F) = k_1k_2k_3
\end{equation*}

\begin{equation*}
	\left(F^{i j}\right)^{-1}=\left(\begin{array}{ccc}
	\frac{\cos \varphi}{k_{1}} & -\frac{\sin \varphi}{k_{1}} & 0 \\
	\frac{\sin \varphi}{k_{2}} & \frac{\cos \varphi}{k_{2}} & 0 \\
	0 & 0 & \frac{1}{k_{3}}
	\end{array}\right)
\end{equation*}


\section{Тензоры деформации Коши-Грина и Альманси}

\subsection{Тензоры Коши-Грина}

\subsubsection*{Правый тензор Коши-Грина}

\begin{align*}
	\left.C=\frac{1}{2}\left(F^{T} F-E\right)&=\frac{1}{2}\left(\left(\begin{array}{ccc}
	k_{1}^{2} & 0 & 0 \\
	0 & k_{2}^{2} & 0 \\
	0 & 0 & k_{3}^{2}
	\end{array}\right)-\left(\begin{array}{ccc}
	1 & 0 & 0 \\
	0 & 1 & 0 \\
	0 & 0 & 1
	\end{array}\right)\right)= \\
	&= \frac{1}{2}\left(\begin{array}{ccc}
	k_{1}^{2}-1 & 0 & 0 \\
	0 & k_{2}^{2}-1 & 0 \\
	0 & 0 & k_{3}^{2}-1
	\end{array}\right)
\end{align*}





\subsubsection*{Левый тензор Коши-Грина}

\begin{equation*}
	\begin{array}{l}
	J=\frac{1}{2}\left(F \cdot F^{T}-E\right)=\\
		&=\frac{1}{2} \left( \left(\begin{array}{ccc}
		k_{1}^{2} \cos ^{2} \varphi+k_{2}^{2} \sin ^{2} \varphi & -k_{1}^{2} \cos \varphi \sin \varphi+k_{2}^{2} \sin \varphi \cos \varphi & 0 \\
		-k_{1}^{2} \cos \varphi \sin \varphi+k_{2}^{2} \sin \varphi \cos \varphi & k_{1}^{2} \sin ^{2} \varphi+k_{2}^{2} \cos ^{2} \varphi & 0 \\
		0 & 0 & k_{3}^{2}
		\end{array}\right) - E\right) = \\
		&= \frac{1}{2}\left(\begin{array}{ccc}
		k_{1}^{2} \cos ^{2} \varphi+k_{2}^{2} \sin ^{2} \varphi-1 & -k_{1}^{2} \cos \varphi \sin \varphi+k_{2}^{2} \sin \varphi \cos \varphi & 0 \\
		-k_{1}^{2} \cos \varphi \sin \varphi+k_{2}^{2} \sin \varphi \cos \varphi & k_{1}^{2} \sin ^{2} \varphi+k_{2}^{2} \cos ^{2} \varphi-1 & 0 \\
		0 & 0 & k_{3}^{2}-1
		\end{array}\right)
	\end{array}
\end{equation*}



\subsection{Тензоры Альманзи}

\subsubsection*{Правый тензор альманзи}

\begin{align*}
	\Lambda&=\left(E-F^{-1}\left(F^{-1}\right)^{T}\right)=\frac{1}{2}\left(\begin{array}{ccc}
	1 & 0 & 0 \\
	0 & 1 & 0 \\
	0 & 0 & 1
	\end{array}\right)-\left(\begin{array}{ccc}
	\frac{1}{k_{1}^{2}} & 0 & 0 \\
	0 & \frac{1}{k_{2}^{2}} & 0 \\
	0 & 0 & \frac{1}{k_{3}^{2}}
	\end{array}\right)=\\
	&= \frac{1}{2}\left(\begin{array}{ccc}
	1-\frac{1}{k_{1}^{2}} & 0 & 0 \\
	0 & 1-\frac{1}{k_{2}^{2}} & 0 \\
	0 & 0 & 1-\frac{1}{k_{3}^{2}}
	\end{array}\right)
\end{align*}


\subsubsection*{Левый тензор Альманзи}

\begin{align*}
	A&=\left(E-\left(F^{-1}\right)^{T} F^{-1}\right)=\\
	&= \frac{1}{2}\left(\left(\begin{array}{ccc}
	1 & 0 & 0 \\
	0 & 1 & 0 \\
	0 & 0 & 1
	\end{array}\right)-\left(\begin{array}{ccc}
	\frac{\cos ^{2} \varphi}{k_{1}^{2}}+\frac{\sin ^{2} \varphi}{k_{2}^{2}} & -\frac{\cos \varphi \sin \varphi}{k_{1}^{2}}+\frac{\cos \varphi \sin \varphi}{k_{2}^{2}} & 0 \\
	-\frac{\cos \varphi \sin \varphi}{k_{1}^{2}}+\frac{\cos \varphi \sin \varphi}{k_{2}^{2}} & \frac{\sin ^{2} \varphi}{k_{1}^{2}}+\frac{\cos ^{2} \varphi}{k_{2}^{2}} & 0 \\
	0 & 0 & \frac{1}{k_{3}^{2}}
	\end{array}\right)\right) = \\
	&= \frac{1}{2}\left(\begin{array}{ccc}
	1-\frac{\cos ^{2} \varphi}{k_{1}^{2}}-\frac{\sin ^{2} \varphi}{k_{2}^{2}} & \frac{\cos \varphi \sin \varphi}{k_{1}^{2}}-\frac{\cos \varphi \sin \varphi}{k_{2}^{2}} & 0 \\
	\frac{\cos \varphi \sin \varphi}{k_{1}^{2}}-\frac{\cos \varphi \sin \varphi}{k_{2}^{2}} & 1-\frac{\sin ^{2} \varphi}{k_{1}^{2}}-\frac{\cos ^{2} \varphi}{k_{2}^{2}} & 0 \\
	0 & 0 & 1-\frac{1}{k_{3}^{2}}
	\end{array}\right)
\end{align*}


\end{document}