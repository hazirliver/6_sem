\documentclass[12 pt]{article}  
\usepackage{cmap}
\usepackage[utf8]{inputenc}
\usepackage[T1,T2A]{fontenc}
\usepackage[english,russian]{babel}      	%sets the font to 12 pt and says this is an article (as opposed to book or other documents)
\usepackage{amssymb}				
\usepackage{mathtools}	% packages to get the fonts, symbols used in most math

%\usepackage{setspace}               		% Together with \doublespacing below allows for doublespacing of the document

\oddsidemargin=-0.5cm                 	% These three commands create the margins required for class
\setlength{\textwidth}{6.5in}         	%
\addtolength{\voffset}{-20pt}        		%
\addtolength{\headsep}{25pt}           	%

\newcommand*{\QEDB}{\hfill\ensuremath{\blacksquare}}%
\newcommand*{\QEDA}{\hfill\ensuremath{\square}}%


\pagestyle{myheadings}                           	% tells LaTeX to allow you to enter information in the heading
\markright{Sokolov Arseny\hfill \today \hfill} 	% put your name instead of Murphy Waggoner 
% and put the proposition number from the book
% LaTeX will put your name on the left, the date the paper 
% is generated in the middle 
% and a page number on the right



\newcommand{\eqn}[0]{\begin{array}{rcl}}%begin an aligned equation - allows for aligning = or inequalities.  Always use with $$ $$
	\newcommand{\eqnend}[0]{\end{array} }  	%end the aligned equation

\newcommand{\qed}[0]{$\square$}        	% make an unfilled square the default for ending a proof

%\doublespacing                         	% Together with the package setspace above allows for doublespacing of the document

\begin{document}												% end of preamble and beginning of text that will be printed
\textbf{Вариант 3.} $g(u_1,u_2) = u_1^2+5u_2^2$
\section{Градиентный метод дробления шага.}
$$x_0=(1,2); \quad \tilde{\alpha}=\varepsilon=\delta=0.1$$

$$g(x_0) = 1^2+5^2\cdot2^2=21$$

$$ g'(u_1,u_2) = (2u_1,10u_2) $$

$$ g'(x_0) = (2,20) \neq (0,0) $$	

$$x_1 = (1,2) - 0.1(2,20) = (0.8,0)$$

$$g(x_1) = 0.64+0 = 0.64$$

Получаем, что 

$$g(x_1)=(1.6, 0) \neq (0,0)$$

Далее

$$g'(x_1) = (1.6,0) \neq (0,0)$$

$$x_2 = (0.8,0) - 0.1(1.6,0) = (0.64,0)$$

$$g(x_2) = 0.4096+0 = 0.4906$$

$$g(x_2) = 0.4096 < 0.64 = g(x_1)$$

Приближенное значение точки минимума возьмём $(0.4096,0)$, тогда минимум функции $m_* \approx 0.4096$.

\newpage

\section{Градиентный метод наискорейшего спуска}
	
$$x_0=(1,2); \quad \delta=0.1$$	
	
$$\varphi_0(\alpha) = g(x_0-\alpha g'(x_0))$$

$$ g'(u_1,u_2) = (2u_1,10u_2) $$

$$ g'(x_0) = (2,20) \neq (0,0) $$
	
$$\varphi_0(\alpha) = g((1,2)-\alpha (2u_1,10u_2)) = g(1-2\alpha, 2-2\alpha) = (1-2\alpha)^2 + 5(2-2\alpha)^2 = 2004\alpha^2-404\alpha+21$$

Минимум функции $\varphi_0(\alpha)$ достигается при $\alpha = \frac{404}{2\cdot 2004} \approx 0.1$. Тогда:

$$x_1 = (1,2) - 0.1(2,20) = (0.8,0)$$

$$g(x_1) = 0.64$$

$$g'(x_1)=(1.6,0)$$

Далее

$$\varphi_1(\alpha) = g(x_1-\alpha g'(x_1)) = g((0.8,0) - \alpha (1.6,0)) = (0.8-1.6\alpha)^2 = 2.56\alpha^2-2.56\alpha+0.64$$

Минимум $\varphi_1(\alpha)$ Достигается при $\alpha = \frac{1}{2}$. Тогда минимум функции $m_* = 0$.

\newpage

\section{Метод сопряжённых направлений}

$$u_0=(1,2)$$

$$Q(u) = Q(u_1,u_2) = u_1^2+5u_2^2$$

$$Q(u) = Q(u_1,u_2) = \frac{1}{2} \langle A u, u\rangle-\langle b, u\rangle$$

$$Q'(u)  = Q'(u_1,u_2) = Au-b,$$

где 

\begin{equation*}
	A=\left(\begin{array}{ll}
	2 & 0 \\
	0 & 10
	\end{array}\right), b=\left(\begin{array}{l}
	0 \\
	0
	\end{array}\right)
\end{equation*}


$$p_0 = Q'(u_1,u_2) = Au-b = \left(\begin{array}{ll}
2 & 0 \\
0 & 10
\end{array}\right) \left(\begin{array}{l}
1 \\
2
\end{array}\right) - \left(\begin{array}{l}
0 \\
0
\end{array}\right) = \left(\begin{array}{l}
2 \\
20
\end{array}\right) \neq \left(\begin{array}{l}
0 \\
0
\end{array}\right)$$


Положим $u_1=u_0-\alpha_0p_0 = \left(\begin{array}{l}
1 \\
2
\end{array}\right) - \alpha_0 \left(\begin{array}{l}
2 \\
20
\end{array}\right)$,


\begin{equation*}
	\alpha_{0}=\frac{\left\langle Q^{\prime}\left(u_{0}\right), p_{0}\right\rangle}{\left\langle A p_{0}, p_{0}\right\rangle}=\frac{\left\langle\left(\begin{array}{c}
		2 \\
		20
		\end{array}\right),\left(\begin{array}{c}
		2 \\
		20
		\end{array}\right)\right\rangle}{\left\langle\left(\begin{array}{cc}
		2 & 0 \\
		0 & 10
		\end{array}\right)\left(\begin{array}{c}
		2 \\
		20
		\end{array}\right),\left(\begin{array}{c}
		2 \\
		20
		\end{array}\right)\right\rangle}=\frac{\left\langle\left(\begin{array}{c}
		2 \\
		20
		\end{array}\right),\left(\begin{array}{c}
		2 \\
		20
		\end{array}\right)\right\rangle}{\left\langle\left(\begin{array}{c}
		4 \\
		200
		\end{array}\right),\left(\begin{array}{c}
		2 \\
		20
		\end{array}\right)\right\rangle}=\frac{404}{4008} \approx 0.1
\end{equation*}


$$u_1 = \left(\begin{array}{l}
1 \\
2
\end{array}\right) - 0.1 \left(\begin{array}{l}
2 \\
20
\end{array}\right) = \left(\begin{array}{l}
0.8 \\
0
\end{array}\right)$$


$$Q'(u_1) = Au_1-b = \left(\begin{array}{cc}
2 & 0 \\
0 & 10
\end{array}\right)\left(\begin{array}{c}
0.8 \\
0
\end{array}\right) - \left(\begin{array}{c}
0 \\
0
\end{array}\right) = \left(\begin{array}{c}
1.6 \\
0
\end{array}\right)$$


$$p_1 = Q'(u_1) - \beta_0p_0 = \left(\begin{array}{c}
1.6 \\
0
\end{array}\right) - \beta_0 \left(\begin{array}{l}
2 \\
20
\end{array}\right)$$


$$\beta_0 = \frac{\left\langle Ap_0, Q^{\prime}\left(u_{1}\right)\right\rangle}{\left\langle A p_{0}, p_{0}\right\rangle}=\frac{\left\langle\left(\begin{array}{c}
	4 \\
	200
	\end{array}\right),\left(\begin{array}{c}
	0.8 \\
	0
	\end{array}\right)\right\rangle}{\left\langle\left(\begin{array}{c}
	4 \\
	200
	\end{array}\right),\left(\begin{array}{c}
	2 \\
	20
	\end{array}\right)\right\rangle}=\frac{3.2}{4008} \approx 0.0008 $$


$$ p_1 = \left(\begin{array}{c}
1.6 \\
0
\end{array}\right) - 0.0008 \left(\begin{array}{l}
2 \\
20
\end{array}\right) = 
\left(\begin{array}{l}
1.5984 \\
-0.016
\end{array}\right)$$


$$u_2=u_1-\alpha_1p_1 = \left(\begin{array}{c}
0.8 \\
0
\end{array}\right) - \alpha_1 \left(\begin{array}{l}
1.5984 \\
-0.016
\end{array}\right),$$


$$\alpha_1 = \frac{\left\langle Q^{\prime}\left(u_{1}\right), p_{1}\right\rangle}{\left\langle A p_{1}, p_{1}\right\rangle}=\frac{\left\langle\left(\begin{array}{c}
	1.6 \\
	0
	\end{array}\right),\left(\begin{array}{c}
	1.5984 \\
	-0.016
	\end{array}\right)\right\rangle}{\left\langle\left(\begin{array}{c}
	3.1968 \\
	-0.016
	\end{array}\right),\left(\begin{array}{c}
	1.5984 \\
	-0.016
	\end{array}\right)\right\rangle}=\frac{2.5574}{5.1100} \approx 0.5 $$



$$u_2 = \left(\begin{array}{c}
0.8 \\
0
\end{array}\right) - 0.5 \left(\begin{array}{l}
1.5984 \\
-0.016
\end{array}\right) = \left(\begin{array}{l}
0.0008\\
0.008
\end{array}\right)$$

Тогда приблизительная точка минимума $(0.0008,0.008)$, $m_* = 0.04.$

















\end{document}